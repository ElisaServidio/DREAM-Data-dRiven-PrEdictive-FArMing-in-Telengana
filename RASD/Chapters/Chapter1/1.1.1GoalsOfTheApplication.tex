\subsection{Goals of the Application}
\begin{itemize}
    \item \textit {G.1}: \textcolor{red}{Allows policy makers to visualize and analyze data of farmers}
    \item \textit {G.2}: Allows policy makers to identify those farmers who are performing well
    \item \textit {G.3}: Allows policy makers to identify those farmers who are performing badly
    \item \textit {G.4}: Allows policy makers to verify the improvement of farmers who have been already helped by agronomist or good farmers
    \item \textit {G.5}: Allows farmers to visualize data and suggestions relevant to them based on their location and type of production
    \item \textit {G.6}: Allows farmers to insert in the system data about their production and any problem they face
    \item \textit {G.7}: Allows farmers to request for help and suggestions by agronomists and other farmers
    \item \textit {G.8}: Allows farmers to create discussion forums with the other farmers
    \item \textit {G.9}: {\color{orange}  Allows agronomists to receive information about requests for help}
    \item \textit {G.10}: Allows agronomists to answer to requests for help from farmers
    \item \textit {G.11}: \textcolor{red}{Allows agronomists to visualize data concerning best performing farmers in the mandal}
    \item \textit {G.12}: \textcolor{red}{Allows agronomists to visualize weather forecasts in the mandal}
    \item \textit {G.14}: Allows agronomists to visualize a daily plan to visit farms in the mandal
    \item \textit {G.15}: Allows agronomists to update a daily plan to visit farms in the mandal
    \item \textit {G.16}: Allows agronomists to confirm the execution of the daily plan at the end of each day 
    \item \textit {G.17}: Allows agronomists to specify the deviations from the daily plan at the end of the day

\end{itemize}