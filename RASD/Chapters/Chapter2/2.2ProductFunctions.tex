\section{Product functions}

\subsection{Registration}
In order to use the application all the actors (policy makers, agronomists and farmers) must be registered. The registration of an account can be performed in two ways: through the Web Application on a computer or through the mobile application on a smartphone.

In the case of the Web application, the farmer opens the web application and clicks on the "Register" button on the homepage. At this point he is redirected to the registration page where the farmer is asked to select which category of users he belongs to: "Policy Maker", "Farmer" or "Agronomist".
After selecting the right category he can enter in the predetermined fields the required data which in the case of the farmer are: name, surname, email, password, telephone number and his farm's address. The format of the characters entered is checked so that it complies with the predetermined rules of good formatting. 
If the rules on well-formed text are respected, a "Register now!" button appears. By clicking on it, the farmer is redirected to a new page that suggests to check his mailbox, where he will find an email generated and sent automatically by the system and with which the farmer can confirm his email by clicking on a link. After that the farmer is redirected to the login page in the Web app. The farmer then receives an email confirming his registration to DREAM.\\

Registration through the mobile application takes place in the same way as above. In fact, the farmer opens the application previously downloaded on his device. On the home page he clicks on the "Register" button. At this point he is redirected to the registration page where he can enter his data. The registration process continues as previously explained.\\

Registration takes place in the same way for all three actors, but there are some differences in the required data depending on the account role of the person who is registering. The data that agronomists and policy makers can enter are the same as those required by a farmer, with the exception of the farm's address.
The latter two roles (agronomists and policy makers) must enter an ID code assigned to them by the Telangana government to gain access to the system with their specific role after it is validated.
Moreover, only agronomists also have to inserts the mandal they are responsible of.


\subsection{Help Request}
DREAM allows farmers to make help request to the agronomist or even to well performing farmers if specified. The farmer accesses to the page "Help Request" and clicks on "Create a Help Request" button and accesses the corresponding page in which, in addition to the agronomist provided by default, he can select as recipient even well performing farmers of his/her same mandal. Eventually the farmer completes the text form corresponding to the body of the help request. Once finished, he/she confirms clicking on the correlated button "Confirm" and the help request is created. The help requests created by each farmer can be visualized by themselves accessing the page "Help Request" and clicking in the button "My Help Request". If someone replies the farmer visualizes in the same page the answer with the associated question \textit{"Are you satisfied with the answer?"} and the corresponding two possible answers \textit{"Yes"} or \textit{"No"}. Whether he/she is satisfied or not, he/she can:
\begin{enumerate}
    \item close the help request clicking on the button "Yes"
    \item not close it clicking on the button "No", waiting to receive other answers
\end{enumerate}
On the other hand, the users who can reply to a help request are the agronomist and well performing farmers of the same mandal. When a help request is created and they are mentioned among its recipients, they can visualize it in the "Help Request" page , where it is listed among the existing others. For each one there is a button "Answer" which they can click on to reply to the request.\\
Once the help request is solved by the farmer who made it, recipients cannot visualize it anymore. 

\subsection{Discussion Forum}
DREAM allows farmers to open a thread on the discussion forum. The farmer accesses the page "Discussion forum" and clicks on the button "Open a thread". He is redirected to a form which he fills in with topic and text and then he confirms. \\
The farmer can also look for a certain topic: he accesses the page "Discussion forum" and clicks on the button "Find a topic", the system provides him a form to fill in with the topic to be found and eventually it provides the farmer the list of existing associated threads.\\
If he wants to reply to one of them he can click on it and then, after pressing the button "Answer", he is redirected to a form in which he has to fill in the text's field. Eventually he confirms.

\subsection{Daily Plan}
DREAM allows agronomists to update the daily plan. He accesses the page "Daily Plan" and clicks on the button "Update". He is redirected to a form which he fills in with date (day and hour) and farmer and then he confirms. \\
The agronomist can also confirm the daily plan at the end of the day: he accesses the page "Daily Plan" and clicks on the button "confirm". He is redirected to a form with a field called "deviations" to be filled in only if modifications have been applied to the daily plan carried out during the day. Eventually he confirms.

\subsection{Notifications}
The system provides the farmer notifications regarding to: 
\begin{itemize}
    \item suggestions according to  his/her relevant information regarding his/her production and his/her farm's position
    \item new visits scheduled in the daily plan regarding him/her
    \item replies to his/her own help requests not yet solved
    \item replies to his/her own threads opened on the Discussion Forum
\end{itemize}
The agronomist and only well performing farmers are also provided with notifications regarding to new help requests received.\\

The actors involved can view the notification area directly from their homepage by clicking on the notification icon. At this point the actor is able to view all the notifications received in chronological order.

\subsection{Farmer inserts data regarding his production}
The farmer is allowed to enter relevant data about his production. Through his home page he has the possibility to insert new data on his harvest.
The data that the farmer has to enter in the form are: type of product and quantity of product collected. In addition, the farmer is allowed to add notes as well.\\
The system will add the current date of the insertion, the identification of the farmer (the email that uniquely identifies him) and the quantity of water consumed, information obtained thanks to some sensors.
After entering these data, the system will update the score for the farmer taking into account the new parameters entered.
The data entered will be saved in the database.

