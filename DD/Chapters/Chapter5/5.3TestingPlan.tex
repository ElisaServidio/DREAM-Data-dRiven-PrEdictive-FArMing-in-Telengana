\section{Testing Plan}

Once all components have been integrated, programmers could perform \textit{System Testing}.
The aim of this kind of testing is to verify functional and non-functional requirements.
Specifically, \textit{DREAM} will be subjcted to the following tests:

\begin{itemize}
    \item \textbf{Functional Testing:} It ensures that the requirements and the specifications defined in the RASD are properly satisfied by the System. 
    \item \textbf{Regression Testing:} This testing is done to make sure that new code changes should not have side effects on the existing functionalities. It ensures that the old code still works once the latest code changes are done.
    \item \textbf{Performance Testing:} This kind of testing identifies the performance bottlenecks in the system affecting response time, utilization, throughput.
    \item \textbf{Load Testing:} It aims at detecting bugs such as memory leaks, mismanagement of memory and buffer overflows. It also identifies the maximum operating capacity of the application.
    \item \textbf{Stress Testing:} It verifies stability and reliability of software application. The goal is measuring software on its robustness and error handling capabilities under extremely heavy load conditions and ensuring that software doesn't crash under crunch situations.
    \item \textbf{Database Testing:} It ensures data values and information received and stored into database are valid or not. Database testing helps to save data loss, saves aborted transaction data and no unauthorized access to the information.
\end{itemize}